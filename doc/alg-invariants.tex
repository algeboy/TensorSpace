
\section{Invariants of nonassociative algebras}

Converting an algebra to a tensor enables Magma to compute standard invariants of any algebra. 
We note that there are known errors for $\mathbb{R}$ and $\mathbb{C}$ due to the numerical stability of the linear algebra involved in the computations.

\color{blue}
\index{Center}\index{Centre}
{\small \begin{verbatim}
Center(A) : Alg -> Alg
Centre(A) : Alg -> Alg
\end{verbatim} }
\color{black}

Returns the center of the algebra $A$.

\color{blue}
\index{Centroid!algebra}
{\small \begin{verbatim}
Centroid(A) : Alg -> AlgMat
\end{verbatim} }
\color{black}

Returns the centroid of the $K$-algebra $A$ as a subalgebra of $\text{End}_K(A)$.

\color{blue}
\index{LeftNucleus!algebra}\index{RightNucleus!algebra}\index{MidNucleus!algebra}
{\small \begin{verbatim}
LeftNucleus(A) : Alg -> AlgMat
RightNucleus(A) : Alg -> AlgMat
MidNucleus(A) : Alg -> AlgMat
\end{verbatim} }
\color{black}

Returns the nucleus of the algebra $A$ as a subalgebra of the enveloping algebra of right multiplication $\mathcal{R}(A)$.

\color{blue}
\index{DerivationAlgebra!algebra}
{\small \begin{verbatim}
DerivationAlgebra(A) : Alg -> AlgMatLie
\end{verbatim} }
\color{black}

Returns the derivation algebra of the algebra $A$ as a Lie subaglebra of $\text{End}_K(A)$.

%%Example{Name}
\begin{framed}{\bf Example} {\tt Alg\_Invariants}\\
{\small 
We demonstrate how to use these functions to get invariants of nonassociative algebras.
First, we will obtain the derivation Lie algebra of the Octonions, which are of type $G_2$.
\begin{lstlisting}[frame=single,basicstyle=\ttfamily\color{black!30!
teal},backgroundcolor=\color{white!70!gray}]
> A := OctonionAlgebra(GF(7),-1,-1,-1);
> A;
Algebra of dimension 8 with base ring GF(7)
> D := DerivationAlgebra(A);
> D.1;
[0 0 0 0 0 0 0 0]
[0 0 6 0 6 3 2 1]
[0 1 0 3 4 1 1 3]
[0 0 4 0 6 4 2 3]
[0 1 3 1 0 6 2 0]
[0 4 6 3 1 0 6 2]
[0 5 6 5 5 1 0 4]
[0 6 4 4 0 5 3 0]
> Dimension(D);
14
> SemisimpleType(D);
G2
\end{lstlisting}
Now we will show that the left, mid, and right nuclei are all one dimensional. 
All of which are generated by $R_1$, multiplication by $1_A$.
\begin{lstlisting}[frame=single,basicstyle=\ttfamily\color{black!30!
teal},backgroundcolor=\color{white!70!gray}]
> Z := Center(A);
> Z;
Algebra of dimension 1 with base ring GF(7)
> 
> L := LeftNucleus(A);
> L;
Matrix Algebra of degree 8 with 1 generator over GF(7)
> L.1;
[1 0 0 0 0 0 0 0]
[0 1 0 0 0 0 0 0]
[0 0 1 0 0 0 0 0]
[0 0 0 1 0 0 0 0]
[0 0 0 0 1 0 0 0]
[0 0 0 0 0 1 0 0]
[0 0 0 0 0 0 1 0]
[0 0 0 0 0 0 0 1]
> 
> L eq MidNucleus(A);
true
> L eq RightNucleus(A);
true
\end{lstlisting}
}
\end{framed}
