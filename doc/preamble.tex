
\usepackage{minitoc}
\usepackage{stmaryrd}  % \llbracket
\usepackage{wasysym}  %\Leftcircle


\newcommand{\midarrow}{\tikz \draw[-triangle 90] (0,0) -- +(.1,0);}
\newcommand{\midArrow}{\tikz \draw[-triangle 90] (0,0) -- +(.1,0);
					   \tikz \draw[-triangle 90] (.1,0) -- +(.1,0);}


\newcommand*{\udots}{\reflectbox{$\ddots$}}

%\makeatletter
%\providecommand*{\Dashv}{%
%  \mathrel{%
%    \mathpalette\@Dashv\vDash
%  }%
%}
%\newcommand*{\@Dashv}[2]{%
%  \reflectbox{$\m@th#1#2$}%
%}
%\makeatother

\makeatletter
\DeclareRobustCommand*\cal{\@fontswitch\relax\mathcal}
\makeatother

\makeatletter
\DeclareRobustCommand*\frak{\@fontswitch\relax\mathfrak}
\makeatother

\newcommand{\onto}{\twoheadrightarrow}

%\DeclareMathAlphabet{\mathpzc}{OT1}{pzc}{m}{it}

% Create chapter autors
\newcommand\chapterauthor[2]{\authortoc{#1}{#2}\printchapterauthor{#1}{#2}}
\makeatletter
\newcommand{\printchapterauthor}[2]{%
  {\parindent0pt\vspace*{-25pt}%
  \linespread{1.1}\large\scshape#1 \\ \footnotesize#2%
  \par\nobreak\vspace*{35pt}}
  \@afterheading%
}
\newcommand{\authortoc}[2]{%
  \addtocontents{toc}{\vskip-10pt}%
  \addtocontents{toc}{%
    \protect\contentsline{chapter}%
    {\hskip1em\mdseries\scshape\protect\scriptsize#1\quad \tiny#2}{}{}}
  \addtocontents{toc}{\vskip5pt}%
}
\makeatother


%-----------------------------------------------------
%       Standard theoremlike environments.
%-----------------------------------------------------
%% \theoremstyle{plain} %% This is the default
\numberwithin{equation}{section}
\newtheorem{thm}{Theorem}
\newtheorem*{thm*}{Theorem}
\newtheorem{mainthm}{Theorem}
\renewcommand*{\themainthm}{\Alph{mainthm}}
\newtheorem{lem}[equation]{Lemma}
\newtheorem{prop}[equation]{Proposition}
\newtheorem{prob}[equation]{Problem}
\newtheorem{lemma}[equation]{Lemma}
\newtheorem{ex}[equation]{Example}
\newtheorem{coro}[equation]{Corollary}
\newtheorem*{coro*}{Corollary}

\theoremstyle{remark}
\newtheorem*{remark*}{Remark}
\newtheorem{remark}[equation]{Remark}

\theoremstyle{definition}
\newtheorem{defn}[equation]{Definition}

\numberwithin{figure}{section}
\numberwithin{table}{section}

%--------------------------------------------------
%       Item references.
%-------------------------------------------------- 
\newcommand{\exref}[1]{Ex\-am\-ple \ref{#1}}
\newcommand{\thmref}[1]{Theo\-rem \ref{#1}}
\newcommand{\defref}[1]{Def\-i\-ni\-tion \ref{#1}}
\newcommand{\eqnref}[1]{(\ref{#1})}
\newcommand{\secref}[1]{Sec\-tion \ref{#1}}
\newcommand{\lemref}[1]{Lem\-ma \ref{#1}}
\newcommand{\propref}[1]{Prop\-o\-si\-tion \ref{#1}}
\newcommand{\corref}[1]{Cor\-ol\-lary \ref{#1}}
\newcommand{\figref}[1]{Fig\-ure \ref{#1}}
\newcommand{\conjref}[1]{Con\-jec\-ture \ref{#1}}
\newcommand{\remref}[1]{Re\-mark \ref{#1}}
\newcommand{\probref}[1]{Prob\-lem \ref{#1}}

% Divides, does not divide.
\providecommand{\divides}{\mid}
\providecommand{\ndivides}{\nmid}
% Normal subgroup or equal.
\providecommand{\normaleq}{\unlhd}
% Normal subgroup.
\providecommand{\normal}{\lhd}
\providecommand{\rnormal}{\rhd}
\providecommand{\union}{\cup}
\providecommand{\bigunion}{\bigcup}
\providecommand{\intersect}{\cap}
\providecommand{\bigintersect}{\bigcap}


%% Homotopism arrows.
\newcommand{\ddd}{\downarrow\downarrow\downarrow}
\newcommand{\ddu}{\downarrow\downarrow\uparrow}
\newcommand{\dde}{\downarrow\downarrow\|}
\newcommand{\dud}{\downarrow\uparrow\downarrow}
\newcommand{\duu}{\downarrow\uparrow\uparrow}
\newcommand{\due}{\downarrow\uparrow\|}
\newcommand{\ded}{\downarrow\|\downarrow}
\newcommand{\deu}{\downarrow\|\uparrow}
\newcommand{\dee}{\downarrow\|\|}
%%
\newcommand{\udd}{\uparrow\downarrow\downarrow}
\newcommand{\udu}{\uparrow\downarrow\uparrow}
\newcommand{\ude}{\uparrow\downarrow\|}
\newcommand{\uud}{\uparrow\uparrow\downarrow}
\newcommand{\uuu}{\uparrow\uparrow\uparrow}
\newcommand{\uue}{\uparrow\uparrow\|}
\newcommand{\ued}{\uparrow\|\downarrow}
\newcommand{\ueu}{\uparrow\|\uparrow}
\newcommand{\uee}{\uparrow\|\|}
%%
\newcommand{\edd}{\|\downarrow\downarrow}
\newcommand{\edu}{\|\downarrow\uparrow}
\newcommand{\ede}{\|\downarrow\|}
\newcommand{\eud}{\|\uparrow\downarrow}
\newcommand{\euu}{\|\uparrow\uparrow}
\newcommand{\eue}{\|\uparrow\|}
\newcommand{\eed}{\|\|\downarrow}
\newcommand{\eeu}{\|\|\uparrow}
\newcommand{\eee}{\|\|\|}

\newcommand{\cev}[1]{\reflectbox{\ensuremath{\vec{\reflectbox{\ensuremath{#1}}}}}}

%--Shortcuts--

%%%%%%%%%%%%%%%%%%%%%%%%%%%%%%%%%%%%%%%

\DeclareMathOperator{\Hol}{Hol}
\DeclareMathOperator{\chr}{char }
\DeclareMathOperator{\trace}{tr~}
\DeclareMathOperator{\rad}{rad }
\DeclareMathOperator{\torrad}{torrad }
\DeclareMathOperator{\Fun}{Fun }
\DeclareMathOperator{\Hom}{hom }
\DeclareMathOperator{\End}{End}
\DeclareMathOperator{\Nil}{Nil }
\DeclareMathOperator{\Ric}{Rich }
\DeclareMathOperator{\Scal}{Scal }
\DeclareMathOperator{\Sym}{Sym }
\DeclareMathOperator{\Alt}{Alt }
\DeclareMathOperator{\Her}{Her }
\DeclareMathOperator{\Adj}{Adj }
\DeclareMathOperator{\Der}{Der }
\DeclareMathOperator{\Spin}{Spin }
\DeclareMathOperator{\JSpin}{JSpin }
\DeclareMathOperator{\GL}{GL}
\DeclareMathOperator{\PGL}{PGL}
\DeclareMathOperator{\SL}{SL}
\DeclareMathOperator{\Sp}{Sp}
\DeclareMathOperator{\GO}{GO}
\DeclareMathOperator{\GU}{GU}
\DeclareMathOperator{\GF}{GF}
\DeclareMathOperator{\Gal}{Gal }
\DeclareMathOperator{\Gr}{Gr}

% Lie algebras.
\DeclareMathOperator{\gl}{\mathfrak{gl}}
%\DeclareMathOperator{\sl}{\mathfrak{sl}}
\DeclareMathOperator{\so}{\mathfrak{so}}
%\DeclareMathOperator{\sp}{\mathfrak{sp}}
\DeclareMathOperator{\Inn}{Inn}
\DeclareMathOperator{\Aut}{Aut}
\DeclareMathOperator{\Inv}{Inv}
\DeclareMathOperator{\Isom}{Isom}
\DeclareMathOperator{\Str}{Str}
\DeclareMathOperator{\Stab}{Stab}
\DeclareMathOperator{\memb}{memb}
\DeclareMathOperator{\Cent}{Cen}
\DeclareMathOperator{\im}{im }
\DeclareMathOperator{\Res}{Res }
\DeclareMathOperator{\Ann}{Ann }
\DeclareMathOperator{\Prob}{Pr}
\DeclareMathOperator{\rank}{rank }
\DeclareMathOperator{\Diag}{Diag }
\DeclareMathOperator{\gr}{gr}
\DeclareMathOperator{\lcm}{lcm}
\DeclareMathOperator{\disc}{disc}
\DeclareMathOperator{\Out}{Out}
\DeclareMathOperator{\out}{out}

\newcommand{\Comm}[1]{{\mathsf{Comm}\textrm{-}{#1}}}
\newcommand{\Set}{{\mathsf{Set}}}

\DeclareMathOperator{\ad}{ad}

\newcommand{\M}{\mathbb{M}}
\newcommand{\cond}[2]{\overset{#1}{{_{#1}#2}_{#1}}}

\newcommand{\Bi}{\mathsf{Bi} }
\newcommand{\Grp}{\mathsf{Grp} }

\newcommand{\lt}[1]{{#1}^{\Lsh}}
\newcommand{\rt}[1]{{#1}^{\Rsh}}
\newcommand{\lr}[1]{{#1}^{\Lsh\Rsh}}
\newcommand{\ct}[1]{{#1}^{\uparrow}}
\newcommand{\perpsum}{\perp} %% \obot is prefered and uses mathabx
\newcommand{\op}{\circ}

\newcommand{\CC}{\mathcal{C}}
\newcommand{\LL}[1]{\mathcal{L}_{#1}}
\newcommand{\RR}[1]{\mathcal{R}_{#1}}
\newcommand{\MM}[1]{\mathcal{M}_{#1}}
\newcommand{\LMR}{\mathcal{LMR}}
\newcommand{\sprod}{\Pi\,}

%\usepackage{ifsym} % \TriangleRight
\newcommand{\trip}[3]{{_{#2}^{#1}} #3}

\newcommand{\lversor}{\,\reflectbox{\ensuremath{\oslash}}\,}%\obackslash}
\newcommand{\rversor}{\oslash}

\newcommand{\bm}{*}
\newcommand{\bmto}{\rightarrowtail}

%  TUPLES===============
%\usepackage[T1]{fontenc}% http://ctan.org/pkg/fontenc
\usepackage{aeguill}
\newcommand{\mm}[1]{\langle #1\rangle}
\newcommand{\la}{\langle}
\newcommand{\ra}{\rangle}
\newcommand{\lga}{\textnormal{\guillemotleft}}
\newcommand{\rga}{\textnormal{\guillemotright}}
\makeatletter
\newsavebox{\@brx}
\newcommand{\lla}[1][]{\savebox{\@brx}{\(\m@th{#1\langle}\)}%
  \mathopen{\copy\@brx\kern-0.5\wd\@brx\usebox{\@brx}}}
\newcommand{\rra}[1][]{\savebox{\@brx}{\(\m@th{#1\rangle}\)}%
  \mathclose{\copy\@brx\kern-0.5\wd\@brx\usebox{\@brx}}}
\makeatother

\newcommand{\bbinom}[2]{\begin{bmatrix} #1\\ #2 \end{bmatrix}}

\newcommand{\lsa}{\sphericalangle}
\newcommand{\rsa}{\reflectbox{\ensuremath{\sphericalangle}}}
\newcommand{\llb}{\llbracket}
\newcommand{\rrb}{\rrbracket}
\newcommand{\llp}{\llparenthesis}
\newcommand{\rrp}{\rrparenthesis}

%\newcommand{\TSat}[3]{#1_{\vDash #2(#3)}}
%\newcommand{\PSat}[3]{{_{#1\vDash}{#2}}_{#3}}
%\newcommand{\OpSat}[3]{{_{#1\vDash #2}#3}}

\newcommand{\bra}[1]{\la #1|}
\newcommand{\ket}[1]{| #1\ra}
\newcommand{\comp}[1]{\bar{#1}}
\newcommand{\widecomp}[1]{\overline{#1}}


%========== NICE HEBREW CHARACTERS  =====================

\usepackage{cjhebrew}
\DeclareFontFamily{U}{rcjhbltx}{}
\DeclareFontShape{U}{rcjhbltx}{m}{n}{<->rcjhbltx}{}
\DeclareSymbolFont{hebrewletters}{U}{rcjhbltx}{m}{n}

% remove the definitions from amssymb
\let\aleph\relax\let\beth\relax
\let\gimel\relax\let\daleth\relax

\DeclareMathSymbol{\aleph}{\mathord}{hebrewletters}{39}
\DeclareMathSymbol{\beth}{\mathord}{hebrewletters}{98}\let\bet\beth
\DeclareMathSymbol{\gimel}{\mathord}{hebrewletters}{103}
\DeclareMathSymbol{\daleth}{\mathord}{hebrewletters}{100}\let\dalet\daleth

\DeclareMathSymbol{\lamed}{\mathord}{hebrewletters}{108}
\DeclareMathSymbol{\mem}{\mathord}{hebrewletters}{109}\let\mim\mem
\DeclareMathSymbol{\ayin}{\mathord}{hebrewletters}{96}
\DeclareMathSymbol{\tsadi}{\mathord}{hebrewletters}{118}
\DeclareMathSymbol{\qof}{\mathord}{hebrewletters}{114}
\DeclareMathSymbol{\shin}{\mathord}{hebrewletters}{152}
\DeclareMathSymbol{\waw}{\mathord}{hebrewletters}{119}
\DeclareMathSymbol{\vavv}{\mathord}{hebrewletters}{119}
\DeclareMathOperator{\vav}{\ensuremath{\vavv}\,}
